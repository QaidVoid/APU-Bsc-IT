\documentclass[12pt]{report}
\usepackage{mathtools}
\usepackage{amsfonts}
\usepackage{amsmath}
\usepackage{array}
\usepackage{colortbl}
\usepackage{pifont}
\usepackage{fontspec}
\setmainfont{Times New Roman}

\title{Probability and Statistical Modeling}
\date{}
\author{QaidVoid}

% Preamble
\begin{document}

\maketitle

\chapter{Concept of Probability}

\section{Introduction}
Probability is the likelihood or chance of something happening. \\
In an experiment in which all outcomes are equally likely, the probability of an event E is,

$P(E) = \frac{\text{number of favourable outcomes}}{\text{total number of outcomes}} = \frac{n(E)}{n(S)}$

\begin{itemize}
  \item[\ding{226}] $0 \le P(E) \le 1$
  \item[\ding{226}] $P(E) = 0$ is an impossible event
  \item[\ding{226}] $P(E) = 1$ is a certain event
  \item[\ding{226}] Sum of all probabilities of an experiment must total 1
  \item[\ding{226}] P(E does not occur) = 1 - P(E) (Complementary Probability)
\end{itemize}

\section{Terminologies}
\begin{itemize}
  \item[\ding{226}] Statistical event
    \subitem It is defined as any subset of the given outcome set that is of interest
  \item[\ding{226}] Statistical experiment
    \subitem It is described as any situation, specially set up or occuring naturally, which can be performed, enacted or otherwise considered in order to gain useful information
  \item[\ding{226}] Sample Space
    \subitem Sample space is the collection of all possible outcomes \\
    Examples: 6 faces of a die, 52 cards of a bridge deck
  \item[\ding{226}] Complement Rule
    \subitem The complement of an event E is the collection of all possible elementary events not contained in event E. The complement of event E is represented by $\overline{E}$.
    \begin{align*}
      P(\overline{E}) & = 1 - P(E) \\
      or, P(E) + P(\overline{E}) & = 1
    \end{align*}
\end{itemize}

\section{Events}
\begin{itemize}
  \item[\ding{226}] Combined Events
    \subitem $P(A \cup B) = P(A) + P(B) - P(A \cap B)$
  \item[\ding{226}] Mutually Exclusive Events
    \subitem Two events of the same experiment are said to be mutually exclusive if their respective events do not overlap.
    \subitem\ding{169} If $E_{1}$ occurs, then $E_{2}$ cannot occur
    \subitem\ding{169} If $E_{1}$ and $E_{2}$ have no common elements
  \item[\ding{226}] Not mulually exclusive
    \subitem If two or more events occur at one time.
  \item[\ding{226}] Independent events
    \subitem Two events are said to be independent if the occurance (or not) of one of the events will in no way affect the occurence (or not) of other.
    \subitem Alternatively, two events that are definde on two physically different experiments are said to be independent.

    \begin{align*}
      E_{1} & = \text{heads on one flip of fair coin} \\
      E_{2} & = \text{heads on second flip of same coin}
    \end{align*}
    Result of second flip does not depend on the result of the first flip.
  \item[\ding{226}] Dependent events
    \begin{align*}
      E_{1} & = \text{rain forecasted on the news} \\
      E_{2} & = \text{take umbrella to work}
    \end{align*}
    Probability of the second event is affected by the occurence of the first event.
  \item[\ding{226}] Conditional event
    \subitem One of the outcomes of which is influenced by the outcomes of another event.
    \subitem If A and B are two events not necessarily from the same experiment, then the conditional probability that A occurs, given that B has already occured, is written,
    \begin{align*}
      P(\text{A, given B}) & = P(A \mid B) = \frac{P(\text{A and B})}{P(B)} \\
      P(\text{B, given A}) & = P(B \mid A) = \frac{P(\text{A and B})}{P(A)}
    \end{align*}
\end{itemize}

\section{Rules of Probability}
\begin{itemize}
  \item[\ding{226}] Addition Rule
    \subitem If A and B are not mutually exclusive events, then
    \begin{align*}
      P(A \cup B) & = P(A) + P(A) - P(A \cap B) \\
      P(A \cup B \cup C) & = P(A) + P(B) + P(C) - P(A \cap B) \\ & - P(A \cap C) - P(B \cap C) + P(A \cap B \cap C)
    \end{align*}
    \subitem If A and B are mutually exclusive events, then
    \begin{align*}
      P(A \cup B) & = P(A) + P(B) \\
      P(A \cup B \cup C) & = P(A) + P(B) + P(C)
    \end{align*}
  \item[\ding{226}] Multiplication Rule
    \subitem If A and B are two events, then the probability of P(A and B), i.e, probability that A and B occur can be calculated as below:
    \subitem Probabilities under conditions of statistical independence
    \begin{align*}
      P(A \cap B) = P(A) \times P(B)
    \end{align*}
    \subitem Probabilities under conditions of statistical dependence
    \begin{align*}
      P(A \cap B) = P(A) \times P(B \setminus A)
    \end{align*}
    \subitem Probability of event B given that A has occured
    \begin{center}
      $P(B \mid A) = \frac{P(\text{A and B})}{P(A)}$
    \end{center}
\end{itemize}

\section{Tree Diagram}
It augments the fundamental principle of counting by exhibiting all possible outcomes of a sequence of events where each event can occur in a finite number of ways.

\section{Contingency Tables}
\begin{itemize}
  \item[\ding{226}] A table used to classify sample observations according to two or more identifiable characteristics
  \item[\ding{226}] It is a cross tabulation that simultaneously summarizes two variables of interest and their relationship
  \item[\ding{226}] Example:
    \subitem A survey of 150 adults classified each as to gender and the number of movies attended last month. Each respondent is classified according to two criteria - the number of movies attended and gender.
    \begin{center}
      \begin{tabular}{|c|c|c|c|}
        \hline
        \multicolumn{4}{|c|}{Gender} \\
        \hline
        Movies Attended & Men & Women & Total \\
        \hline
        0 & 20 & 40 & 60 \\
        \hline
        1 & 40 & 30 & 70 \\
        \hline
        2 or more & 10 & 10 & 20 \\
        \hline
        Total & 70 & 80 & 150 \\
        \hline
      \end{tabular}
    \end{center}
\end{itemize}

\section{Posterior Probability}
\begin{itemize}
  \item[\ding{226}] Bayes Theorem
    \subitem\ding{227} It is a formula which can be thought of as 'reversing' conditional probability. That is, it finds a conditional probability (A|B) given, among other things, its inverse (B|A).
    \subitem\ding{227} If A and B are two events of an experiment, then
    \begin{align*}
      P(A|B) = \frac{P(A) \times P(B|A)}{P(B)}
    \end{align*}
\end{itemize}

\section{Examples}
\subsection*{Example 1.1}
A group of 20 university students contain eight who are in the first year of study. A student is picked at random. Find the probability that the student is not in the first year of study. \\[20pt]
Given, \\
$n(S) = 20$ \\
$n(E) = 20 - 8 = 12$ \\
Now, \\
$P(E) = \frac{n(E)}{n(S)} = \frac{12}{20} = \frac{3}{5}$

\subsection*{Example 1.2}
If a card is drawn from a deck of playing cards, what is the probability of getting a red or an ace? \\[20pt]
Sample Space, $n(S) = 52$ \\
No. of red cards, $n(\text{Red}) = 26$ \\
No. of ace, $n(\text{Ace}) = 4$ \\
No. of red aces, $n(\text{Red and Ace}) = 2$ \\
Now, \\
\begin{align*}
  P (\text{Red or Ace}) & = P(\text{Red}) + P(\text{Ace}) - P(\text{Red and Ace}) \\
  & = \frac{26}{52} + \frac{4}{52} - \frac{2}{52} \\
  & = \frac{28}{52} \\
  & = \frac{7}{13}
\end{align*}

\subsection*{Example 1.3}
In a class of 20 children, 4 of the boys and 3 of the eleven girls are in the athletics team. A child from the class is chosen at random. Find the probability that the child chosen is \\
(a) in the athletics team \\
(b) a girl \\
(c) a girl member of the athletics team \\
(d) a girl member or in the athletics team \\[20pt]
Given, \\
\indent Total number of children, $n(S) = 20$ \\
\indent Number of boys, $n(B) = 4$ \\
\indent Number of girls, $n(G) = 3$ \\[20pt]
(a) \\
\indent Probability that the children is in the athletics team, \\
\indent $n(A) = n(B) + n(G) = 4 + 3 = 7$ \\
\indent $P(A) = \frac{n(E)}{n(S)} = \frac{7}{20}$ \\[20pt]
(b) \\
\indent Probability that the children is a girl, \\
\indent $P(G) = \frac{n(E)}{n(S)} = \frac{11}{20}$ \\[20pt]
(c) \\
\indent Probability that the children is a girl member of the athletics team, \\
\indent $P(G \cap A) = \frac{3}{20}$ \\[20pt]
(d) \\
\indent Probability that the children is a girl member or in the athletics team, \\
\indent $P(G \cup A) = P(A) + P(G) - P(G \cap A)$ \\[5pt]
\indent $P(G \cup A) = \frac{7}{20} + \frac{11}{20} - \frac{3}{20}$ \\[5pt]
\indent $P(G \cup A) = \frac{15}{20} = \frac{3}{4}$

\subsection*{Example 1.4}
In a race in which there are no dead heats, the probability that John wins is 0.3, the probability that Paul wins is 0.2 and the probabiity that Mark wins is 0.4. Find the probability that \\
(a) John or Mark wins \\
(b) John or Paul or Mark wins \\
(c) someone else wins \\[20pt]
Given, \\
\indent Probability that John wins, P(J) = 0.3 \\
\indent Probability that Paul wins, P(P) = 0.2 \\
\indent Probability that Mark wins, P(M) = 0.4 \\[20pt]
(a) \\
\indent Probability that John or Mark wins, \\
\indent $P(J \cup M) = P(J) + P(M)$ \\
\indent $P(J \cup M) = 0.3 + 0.4$ \\
\indent $(J \cup M) = 0.7$ \\[20pt]
(b) \\
\indent Probability that John or Paul or Mark wins, \\
\indent $P(J \cup P \cup M) = P(J) + P(P) + P(M)$ \\
\indent $P(J \cup P \cup M) = 0.3 + 0.2 + 0.4$ \\
\indent $P(J \cup P \cup M) = 0.9$ \\[20pt]
(c) \\
\indent Probability that someone else wins, \\
\indent $P(E) = 1 - P(J \cup P \cup M)$ \\
\indent $P(E) = 1 - 0.9$ \\
\indent $P(E) = 0.1$

\subsection*{Example 1.5}
A card is drawn from a bag containing 5 red cards numbered 1 to 5 and 3 green cards numbered 1 to 3. Find the probability that the card is \\
(a) a green card or a red card \\
(b) a green card or an even number \\[20pt]
Given, \\
\indent Total number of cards, $n(S) = 8$ \\
\indent No. of red cards, $n(R) = 5$ \\
\indent No. of green cards, $n(G) = 3$ \\
\indent Even numbered cards, $n(E) = 3$ \\[20pt]
(a) \\
\indent Probability that the card is green or red, \\
\indent $P(G \cup R) = P(G) + P(R)$ \\
\indent $P(G \cup R) = \frac{n(G)}{n(S)} + \frac{n(R)}{n(S)}$ \\
\indent $P(G \cup R) = \frac{3}{8} + \frac{5}{8}$ \\
\indent $P(G \cup R) = 1$ \\[20pt]
(b) \\
\indent Probability that the card is green or an even number, \\
\indent $P(G \cup E) = P(G) + P(E)$ \\
\indent $P(G \cup E) = \frac{n(E)}{n(S)} + \frac{n(E)}{n(S)}$ \\[5pt]
\indent $P(G \cup E) = \frac{3}{8} + \frac{3}{8}$ \\[5pt]
\indent $P(G \cup E) = \frac{3}{4}$
\end{document}
